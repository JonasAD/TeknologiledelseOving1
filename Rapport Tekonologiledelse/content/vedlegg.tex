\chapter{Vedlegg}

\section{Mail}

\subsection{Svar fra Expert}
Spennende med nye produkter innenfor kaffe, spesielt siden Norden drikker mest kadde pr person i verden.

Produktet i seg selv er innovativt og kan være en mulighet, men utfordringen er nok at en god thermos kopp / eller to-go kopp gjør samme nytten og holder på varmen i flere timer uten at en tilkobling til strøm er nødvendig. Ett annet element dere må ta i betraktning er at kaffen ikke smaker særlig godt etter lang tid så den må jo drikkes innen rimelig tid.

Det som taler for produktet er at det finnes mye gadgets man kan kjøpe til kontorpulten sin, alt fra USB vifter til rare duppedingser som kobles til strøm og har en eller annen funksjon på pulten.
Tar man produktet hakket videre ville jeg sagt en to-og kopp med presskanne funksjonalitet og tilkobling med USB eller induksjon som her. Da kan du lage kaffen direkte på pulten og slipper kaffetrakteren.

\subsection{Svar fra Tilbords}
Ideen er god, men jeg tenker at det lett kan bli litt søl.  Da er det i alle fall viktig at koppen har lokk.  Kanskje en takeawaykopp er svaret?  Det er jo også begrenset hvilke type kaffe du kan lage i en slik kopp.  Hvis formålet er å bruke det som en minivannkoker er det greit.

Hvis det kun er en noe som skal holde kaffen varm tror jeg dette blir et marginalt produkt.  En kopp kaffe drikker du så fort at du ikke trenger å holde den varm. 

\subsection{Svar fra Teknikkmagasinet}
Det är absolut en produkt vi skulle kunna tänka oss att sälja, vi har redan en väldigt likvärdig produkt men utan induktion.
http://www.teknikmagasinet.no/produkter/leker-o-gadgets/usb-gadgets/usb-koppvarmer

Priset är dock lite i högsta laget om man jämför med vad den befintliga modellen kostar.

