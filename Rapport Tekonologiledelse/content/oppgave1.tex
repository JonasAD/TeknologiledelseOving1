\chapter{Brainstorming}
\section{Kaffekopp som varmer kaffen}
Kaffen er en viktig del av mange menneskers hverdag. Kaffekoppen er med oss ved frokosten, på jobb og til kake festlige lag. Et av problemene som oppstår ved drikking av kaffe er at temperaturen er feil. Etter brygging er kaffen for varm til å drikkes og man lar den gjerne stå for å avkjøle den. Og hvor mange ganger opplever man ikke da at man venter for lenge og plutselig er kaffen kald og besk? Eller kanskje man kommer halvveis ned i kaffekoppen, før kaffen er blitt kald og udelikat å drikke.

Det finnes produkter på markedet som prøver å forhindre denne problematikken allerede, blant annet termokopper. Men en termokopp forsinker bare prosessen og er ikke en fullverdig løsning på problemet i våre øyne. Lukker man lokket på koppen med en gang man har helt oppi den varme kaffen er kaffen for varm til å drikkes lenge før den er mulig å drikke. Man ender derfor fort opp med å ha lokket på koppen åpen en stund først for at den skal få kjølt seg litt ned og man risikerer at kaffen blir for kald før man setter på lokket igjen. I tillegg holder slutter ikke temperaturen å synke, prosessen bare forsinkes.

En kaffekopp med et varmeelement som ved hjelp av elektrisitet holder kaffen varm kunne ha vært en mulig løsning på dette problemet. En kaffekopp med innebygget varmelement som drives ved hjelp av en induksjonsplate man kan sette koppen på holder kaffen kontinuerlig varm innen for et predefinert temperaturområde som ansees som "perfekt" drikketemperatur. På den måten vil kaffen være i perfekt temperatur hele tiden og man slipper å helle ut de kalde restene man ofte sitter igjen med på slutten.


\section{Legitimasjon på mobilen}
Legitimasjon er noe som er kjekt å ha med seg uansett hvor enn man ferdes. Enten man skal handle øl i butikken, på ferie eller inn på et utested må man ha legitimasjon tilgjengelig. Skulle man bare ut en kjapp tur, eller hadde man dårlig tid når man dro hjemmefra, kan det være fort gjort å glemme legitimasjonen.

I tillegg er passet for noen eneste tilgjengelige legitimasjon. Man har ikke tatt lappen enda og bankkortet man har har ikke legitimasjon.

I dagens samfunn er mobilen en dings de aller fleste er veldig avhengige av for å utføre dagligdagse oppgaver. Av den grunn har de aller fleste mobilen med seg overalt. Det å kunne ha legitimasjonen med seg på mobilen kan vi se for oss kan være noe folk kan ha nytte av. En slik løsning ser vi for oss at kan være implementert via en app på telefonen.

Ved en slik løsning ville de største utfordringene vært å gjøre legitimasjonen sikker, umulig å kopiere og umulig å forfalske. I tillegg ville det vært en utfordring å få det godkjent som gyldig legitimasjon, og man hadde nok vært avhengig av samarbeid med statlige organisasjoner som politiet eller ulike banker. Vi valgte å ikke gå videre på denne ideen.